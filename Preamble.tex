\setstretch{1.5}

\title{Improved Quantification for Respiratory Gated PET/CT: Data-Driven Algorithms for Respiratory Motion Correction in PET/CT}
\author{Alexander Charles Whitehead}
\supervisors{Prof. Kris Thielemans\\
             Dr Jamie R. McClelland}
\field{Medical Imaging Physics and Medical Image Computing}
\department{Department of Medical Physics and Biomedical Engineering\\
            Institute of Nuclear Medicine}

\maketitle

\setstretch{2.0}

% \epigraph{\textit{``It [is] better to burn than to disappear [...] [It] made me realise that I'd been happy, and that I was happy still [...] for all to be accomplished''}}{Albert Camus, \textit{L'Étranger}}

% \epigraph{\textit{``Every breath you take every move you make [...] I'll be watching you''}}{Sting and The Police, \textit{Every Breath You Take}}

\thispagestyle{empty}
\vspace*{\fill}
\setlength{\epigraphwidth}{0.7\linewidth}
\renewcommand{\epigraphflush}{center}
\renewcommand{\epigraphsize}{\large}
\epigraph{\textit{In dark times,\newline should the stars also go out?}}%
         {``Steban, the Student Communist''\\ \textit{Disco Elysium}}
\vfill

\begin{abstract}
    Respiratory motion is a significant problem when it comes to the quality of \gls{PET} images. Respiratory motion not only causes blurring of lesions, and other features within the lung, but can also cause problems when it comes to applying attenuation correction to the acquired \gls{PET} data. Respiratory motion can be mitigated by gating but this leads to longer acquisition time. Problems related to attenuation correction include the fact that the attenuation map is acquired temporally apart from the \gls{PET} scan, and different instructions are given to the patient for each scan (free breathing vs breath hold). Thus, the \gls{Mu-Map} may not correspond to any position in the \gls{PET} data and certainly won't correspond to all of them. Meaning that if a static \gls{Mu-Map} is used to apply attenuation correction not only will the respiratory motion exist to corrupt the data but so will the mismatch of the \gls{Mu-Map}.

    Motion correction methods exist which incorporate registration in order to attempt to improve upon non-motion corrected results. Often these methods involve separating the \gls{PET} data into bins, where the respiratory motion is minimised within each bin. However, because of the high level of noise and low spatial resolution of \gls{PET} data, few bins are often used, leading to a not insignificant amount of respiratory motion still being present in the resultant images. Additionally, this approach does not solve the mismatch of the \gls{Mu-Map}. Logically, the bin closest to the \gls{Mu-Map} could be used as the reference bin for registration but it is not guaranteed that this bin will be close to the position of the \gls{Mu-Map} and as such artefacts will remain in the final image. Furthermore, registration fails when used on dynamic \gls{PET} data, where the signal from the aorta, at early time points, often leads to misregistration. More complex motion correction methods exist, however, these methods, in general, tend to be more resource intensive in both the sense of computation time and computational resources.

    This thesis focuses on the development of a motion correction method, which seeks to rectify some of the issues above by using respiratory gated data in combination with motion modelling. Firstly, this thesis will present more preliminary results, where the bounds of the problem were found. This includes experiments to discover the effectiveness of different motion correction techniques (focusing on motion modelling) in the case of \gls{TOF} vs \gls{Non-TOF} \gls{PET} data, especially where a high number of bins are used. This thesis will also present work related to attempting to solve the mismatch of the \gls{Mu-Map} by deforming it to the position during the gates. Furthermore, work will also be presented which compares the effects of the reconstruction and \gls{MM} fitting process, including using \gls{MLACF}, to approximate attenuation correction, as well as fitting a \gls{MM} on coarsely binned data and applying it to finely binned data.
    
    Finally, this thesis will also present work on the problems of \gls{DD} \gls{SS} extraction methods applied to dynamic \gls{PET}. A \gls{SS} is imperative to the effectiveness of binning data as well as to \gls{MM} fitting. By presenting work relevant to \gls{SS} extraction from dynamic \gls{PET} this work potentially opens the motion correction methods presented previously to the application of dynamic \gls{PET}.

    This thesis will conclude with a critique of the work presented previously as well as a look to the future of how this work could be improved or further used.
\end{abstract}

\begin{impactstatement}
    \begin{quote}
        
    \end{quote}
\end{impactstatement}

\begin{acknowledgements}
    This research was supported by \gls{GE} Healthcare, the \gls{NIHR} \gls{UCLH} Biomedical Research Centre and the \gls{UCL} \gls{EPSRC} \gls{CDT} in \gls{i4health} grant (EP/L$016478$/$1$).
    
    The software used was partly produced by the \gls{CCP} on \gls{SyneRBI} \gls{UK} \gls{EPSRC} grant (EP/T$026693$/$1$).

    Abschließend möchte ich mich bei Kathi bedanken. Ich würde gerne scherzen, dass es viel einfacher gewesen wäre, meine Doktorarbeit zu beenden, wenn Du nicht bei mir gewesen wärst, weil ich meine ganze Zeit nur mit Dir verbringen möchte. Aber das stimmt nicht (na ja, das mit der Zeit schon). Ich glaube nicht, dass ich meine Arbeit hätte abschließen können, zumindest nicht so glücklich, wie ich es bin, wenn Du nicht gewesen wärst. Obwohl in letzter Zeit so viel passiert ist, findest Du immer noch etwas in Dir, das Dir hilft, jeden Tag besser zu machen als den letzten. Ich glaube wirklich, dass du meine Arbeit öfter gelesen hast als ich selbst. Dafür möchte ich Dir danken. Ich denke, wir können mit Fug und Recht behaupten, dass wir ``in guten wie in schlechten Zeiten'' und ``in Krankheit und Gesundheit'' bereits gemeistert haben. Ich bin gespannt auf ``bis dass der Tod uns''.

    % Finally, I would like to thank Kathi. I wanted to joke that it would have been a lot easier to finish my thesis if you weren't around, because all I want to do is spend all my time on you. However, that's not true (well it is about the time thing, I can't help myself). I don't think I could have finished my thesis, at least not while being as happy as I am, if it wasn't for you. Even though there has been so much going on recently, you still find something in yourself that helps to make every day better than the last. I genuinely think you've read my thesis more times than I have. I really appreciate you. I think we can legitimately say that we've already completed ``for better, for worse'' and ``in sickness and in health'', I'm excited to see about ``till death do us part''. I'd also like to thank Birte for help with the German. Otherwise, with the German I know, the previous paragraph would have been mostly profanity.
    
%     \newpage
%     
%     N\-9\-G\-s\-q\-H\-V\-Z\-3\-x\-l\-a\-t\-z\-d\-p\-i\-y\-e\-r\-q\-9\-8\-q\-l\-i\-G\-G\-V\-u\-p\-x\-1\-A\-F\-a\-u\-7\-9\-2\-9\-l\-l\-V\-j\-I\-N\-C\-u\-E\-F\-7\-O\-w\-o\-y\-L\-0\-e\-C\-Y\-B\-O\-0\-L\-V\-V\-4\-5\-o\-7\-y\-0\-a\-R\-W\-o\-d\-C\-d\-x\-N\-o\-n\-b\-5\-P\-V\-H\-4\-0\-k\-C\-z\-Q\-p\-a\-S\-H\-b\-4\-8\-f\-F\-t\-O\-V\-3\-M\-L\-I\-g\-t\-X\-c\-F\-V\-e\-x\-g\-P\-L\-Z\-1\-q\-l\-8\-7\-o\-n\-a\-N\-F\-a\-A\-F\-F\-C\-K\-p\-i\-7\-b\-1\-S\-e\-4\-R\-s\-I\-A\-6\-L\-t\-6\-N\-m\-k\-q\-w\-4\-W\-c\-h\-P\-9\-T\-a\-T\-I\-N\-u\-B\-L\-A\-H\-3\-2\-6\-s\-H\-Z\-4\-g\-z\-F\-X\-n\-J\-6\-s\-1\-7\-w\-k\-Z\-e\-I\-W\-1\-j\-x\-5\-M\-y\-8\-x\-Y\-H\-S\-F\-e\-g\-U\-R\-V\-K\-z\-l\-t\-x\-S\-F\-K\-E\-m\-E\-V\-h\-s\-x\-a\-w\-M\-C\-G\-8\-p\-z\-7\-c\-8\-o\-1\-E\-L\-o\-H\-L\-m\-p\-x\-V\-P\-U\-l\-F\-M\-d\-c\-i\-v\-0\-Y\-1\-H\-E\-h\-t\-p\-Y\-j\-t\-1\-7\-s\-j\-p\-h\-q\-w\-0\-Q\-Y\-w\-V\-b\-M\-B\-N\-m\-F\-B\-h\-k\-g\-P\-i\-6\-y\-N\-J\-k\-p\-G\-C\-U\-l\-9\-s\-V\-K\-h\-V\-P\-5\-3\-W\-p\-Y\-u\-F\-E\-O\-l\-t\-H\-o\-8\-4\-K\-J\-L\-W\-e\-O\-h\-C\-m\-J\-B\-T\-c\-G\-a\-j\-n\-l\-A\-u\-D\-x\-L\-P\-w\-9\-u\-n\-8\-k\-q\-P\-6\-h\-B\-H\-Y\-q\-L\-5\-w\-n\-2\-b\-Q\-l\-0\-p\-q\-M\-x\-y\-2\-5\-1\-B\-f\-R\-G\-y\-L\-3\-5\-O\-3\-t\-q\-j\-Z\-I\-Q\-9\-w\-f\-4\-H\-s\-J\-r\-i\-u\-9\-F\-p\-O\-Y\-4\-E\-Y\-o\-T\-G\-f\-p\-j\-m\-6\-G\-t\-v\-o\-W\-a\-i\-K\-x\-1\-J\-J\-n\-W\-F\-f\-7\-8\-S\-n\-i\-f\-T\-0\-u\-b\-A\-1\-W\-4\-7\-X\-K\-0\-0\-m\-D\-p\-1\-n\-o\-s\-a\-3\-C\-l
%     
%     1966 1995
\end{acknowledgements}

\preto\fullcite{\AtNextCite{\defcounter{maxnames}{99}}}

\begin{publications}
 \begin{itemize}
    \item \fullcite{Whitehead2023DataComponents}
    
    \item \fullcite{Shahin2023CenTime:Analysis}

    \item \fullcite{Whitehead2023APET}

    \item \fullcite{Whitehead2023DeepPatients}
    
    \item \fullcite{Ovtchinnikov2023SIRFFramework}
    
    \item \fullcite{Brusaferri2022DetectionData}

    \item \fullcite{Whitehead2022PET/CTData}

    \item \fullcite{Whitehead2022DataPCA}

    \item \fullcite{Whitehead2022Pseudo-BayesianPET}
     
    \item \fullcite{Ferrante2022PhysicallyImaging}

    \item \fullcite{Pasca2022SIRF-SuperBuild}
    
    \item \fullcite{Whitehead2021ComparisonMap}
    
    \item \fullcite{Biguri2021SystematicPET}

    \item \fullcite{Ovtchinnikov2021SIRFMachine}
     
    \item \fullcite{Whitehead2020PET/CTFields}
     
    \item \fullcite{Whitehead2019ImpactPET}
    
    \item \fullcite{Efthimiou2019PreliminaryPET}
 \end{itemize}
\end{publications}

\tableofcontents
\listoffigures
\listoftables
\printglossaries
