\setstretch{2.0}
\setstretch{1.5}

\title{Improved Quantification for Respiratory Gated PET/CT: Data-Driven Algorithms for Respiratory Motion Correction in PET/CT}
\author{Alexander Charles Whitehead}
\supervisors{Prof. Kris Thielemans\\
             Dr Jamie R. McClelland}
\field{Medical Imaging Physics and Medical Image Computing}
\department{Department of Medical Physics and Biomedical Engineering\\
            Institute of Nuclear Medicine}

\maketitle

\setstretch{2.0}

% \epigraph{\textit{"It [is] better to burn than to disappear [...] [It] made me realise that I'd been happy, and that I was happy still [...] for all to be accomplished"}}{Albert Camus, \textit{L'Étranger}}

% \epigraph{\textit{"Every breath you take every move you make [...] I'll be watching you"}}{Sting and The Police, \textit{Every Breath You Take}}

\thispagestyle{empty}
\vspace*{\fill}
\setlength{\epigraphwidth}{0.7\linewidth}
\renewcommand{\epigraphflush}{center}
\renewcommand{\epigraphsize}{\large}
\epigraph{\textit{In dark times,\newline should the stars also go out?}}%
         {"Steban, the Student Communist"\\ \textit{Disco Elysium}}
\vfill

\begin{abstract}
    \gls{RM} is a significant problem when it comes to the quality of \gls{PET} images. \gls{RM} not only causes blurring of the diaphragm, and other features within the lung, but can also cause problems when it comes to applying \gls{AC} to the acquired \gls{PET} data. \gls{RM} cannot be directly mitigated due to the long acquisition time of \gls{PET}. Problems related to \gls{AC} include the fact that often the attenuation map is acquired temporally apart from the \gls{PET} scan, and different instructions are given to the patient for each scan (free breathing vs breath hold). Thus, the \gls{Mu-Map} may not correspond to any position in the \gls{PET} data and certainly won't correspond to all of them. Meaning that if a static \gls{Mu-Map} is used to apply \gls{AC} not only will the \gls{RM} exist to corrupt the data but so will the mismatch of the \gls{Mu-Map}.

    \gls{MC} methods exist which incorporate \gls{IR} in order to attempt to improve upon \gls{nMC} results. Often these methods involve separating the \gls{PET} data into bins, where the \gls{RM} is minimised within each bin. However, because of the high level of noise and low spatial resolution of \gls{PET} data, few bins are often used, leading to a not insignificant amount of \gls{RM} still being present in the resultant images. Additionally, this approach does not solve the mismatch of the \gls{Mu-Map}. Logically, the bin closest to the \gls{Mu-Map} could be used as the reference bin for \gls{IR} but it is not guaranteed that this bin will be close to the position of the \gls{Mu-Map} and as such artefacts will remain in the final image. Furthermore, \gls{IR} fails when used on dynamic \gls{PET} data, where the signal from the aorta, at early time points, often leads to misregistration. More complex \gls{MC} methods exist, however, these methods, in general, tend to be more resource intensive in both the sense of computation time and computational resources.

    This thesis focuses on the development of a \gls{MC} method, incorporating \glss{MM}, which seeks to rectify some of the issues above. This thesis will first discuss the background literature surrounding the topic, before moving on to presenting work performed to discover a solution. Firstly, this thesis will present more preliminary results, where the bounds of the problem were found. This includes experiments to discover the effectiveness of different \gls{MC} techniques (focusing on \glss{MM}) in the case of \gls{TOF} vs \gls{nTOF} \gls{PET} data, especially where a high number of bins are used. This thesis will also present work related to attempting to solve the mismatch of the \gls{Mu-Map} by deforming it to the position of the bins. Furthermore, work will also be presented which compares the effects of the reconstruction and \gls{MM} fitting process, including using \gls{MLACF}, to approximate \gls{AC}, as well as fitting a \gls{MM} on coarsely binned data and applying it to finely binned data.
    
    As a tangent, this thesis will also present work on the problems of \gls{DD} \gls{SS} extraction methods applied to dynamic \gls{PET}. A \gls{SS} is imperative to the effectiveness of binning data as well as to \gls{MM} fitting. By presenting work relevant to \gls{SS} extraction from dynamic \gls{PET} this work potentially opens the \gls{MC} methods presented previously to the application of dynamic \gls{PET}.

    This thesis will conclude with a critique of the work presented previously as well as a look to the future of how this work could be improved or further used.
\end{abstract}

\begin{impactstatement}
    \begin{quote}
        
    \end{quote}
\end{impactstatement}

\begin{acknowledgements}
    
    
%     \newpage
%     
%     N\-9\-G\-s\-q\-H\-V\-Z\-3\-x\-l\-a\-t\-z\-d\-p\-i\-y\-e\-r\-q\-9\-8\-q\-l\-i\-G\-G\-V\-u\-p\-x\-1\-A\-F\-a\-u\-7\-9\-2\-9\-l\-l\-V\-j\-I\-N\-C\-u\-E\-F\-7\-O\-w\-o\-y\-L\-0\-e\-C\-Y\-B\-O\-0\-L\-V\-V\-4\-5\-o\-7\-y\-0\-a\-R\-W\-o\-d\-C\-d\-x\-N\-o\-n\-b\-5\-P\-V\-H\-4\-0\-k\-C\-z\-Q\-p\-a\-S\-H\-b\-4\-8\-f\-F\-t\-O\-V\-3\-M\-L\-I\-g\-t\-X\-c\-F\-V\-e\-x\-g\-P\-L\-Z\-1\-q\-l\-8\-7\-o\-n\-a\-N\-F\-a\-A\-F\-F\-C\-K\-p\-i\-7\-b\-1\-S\-e\-4\-R\-s\-I\-A\-6\-L\-t\-6\-N\-m\-k\-q\-w\-4\-W\-c\-h\-P\-9\-T\-a\-T\-I\-N\-u\-B\-L\-A\-H\-3\-2\-6\-s\-H\-Z\-4\-g\-z\-F\-X\-n\-J\-6\-s\-1\-7\-w\-k\-Z\-e\-I\-W\-1\-j\-x\-5\-M\-y\-8\-x\-Y\-H\-S\-F\-e\-g\-U\-R\-V\-K\-z\-l\-t\-x\-S\-F\-K\-E\-m\-E\-V\-h\-s\-x\-a\-w\-M\-C\-G\-8\-p\-z\-7\-c\-8\-o\-1\-E\-L\-o\-H\-L\-m\-p\-x\-V\-P\-U\-l\-F\-M\-d\-c\-i\-v\-0\-Y\-1\-H\-E\-h\-t\-p\-Y\-j\-t\-1\-7\-s\-j\-p\-h\-q\-w\-0\-Q\-Y\-w\-V\-b\-M\-B\-N\-m\-F\-B\-h\-k\-g\-P\-i\-6\-y\-N\-J\-k\-p\-G\-C\-U\-l\-9\-s\-V\-K\-h\-V\-P\-5\-3\-W\-p\-Y\-u\-F\-E\-O\-l\-t\-H\-o\-8\-4\-K\-J\-L\-W\-e\-O\-h\-C\-m\-J\-B\-T\-c\-G\-a\-j\-n\-l\-A\-u\-D\-x\-L\-P\-w\-9\-u\-n\-8\-k\-q\-P\-6\-h\-B\-H\-Y\-q\-L\-5\-w\-n\-2\-b\-Q\-l\-0\-p\-q\-M\-x\-y\-2\-5\-1\-B\-f\-R\-G\-y\-L\-3\-5\-O\-3\-t\-q\-j\-Z\-I\-Q\-9\-w\-f\-4\-H\-s\-J\-r\-i\-u\-9\-F\-p\-O\-Y\-4\-E\-Y\-o\-T\-G\-f\-p\-j\-m\-6\-G\-t\-v\-o\-W\-a\-i\-K\-x\-1\-J\-J\-n\-W\-F\-f\-7\-8\-S\-n\-i\-f\-T\-0\-u\-b\-A\-1\-W\-4\-7\-X\-K\-0\-0\-m\-D\-p\-1\-n\-o\-s\-a\-3\-C\-l
%     
%     1966 1995
\end{acknowledgements}

\preto\fullcite{\AtNextCite{\defcounter{maxnames}{99}}}

\begin{publications}
 \begin{itemize}
    \item \fullcite{Ovtchinnikov2023SIRFFramework}
    
    \item \fullcite{Brusaferri2022DetectionData}
     
    \item \fullcite{Ferrante2022PhysicallyImaging}

    \item \fullcite{Pasca2022SIRF-SuperBuild}
    
    \item \fullcite{Whitehead2021ComparisonMap}
    
    \item \fullcite{Biguri2021SystematicPET}

    \item \fullcite{Ovtchinnikov2021SIRFMachine}
     
    \item \fullcite{Whitehead2020PET/CTFields}
     
    \item \fullcite{Whitehead2019ImpactPET}
    
    \item \fullcite{Efthimiou2019PreliminaryPET}
 \end{itemize}
\end{publications}

\tableofcontents
\listoffigures
\listoftables
\printglossaries
