\setstretch{1.5}

\title{Improved Quantification for Respiratory Gated PET/CT: Data-Driven Algorithms for Respiratory Motion Correction in PET/CT}
\author{Alexander Charles Whitehead}
\supervisors{Prof. Kris Thielemans\\
             Dr Jamie R. McClelland}
\field{Medical Imaging Physics and Medical Image Computing}
\department{Department of Medical Physics and Biomedical Engineering\\
            Institute of Nuclear Medicine}

\maketitle

\setstretch{2.0}

% \epigraph{\textit{``It [is] better to burn than to disappear [...] [It] made me realise that I'd been happy, and that I was happy still [...] for all to be accomplished''}}{Albert Camus, \textit{L'Étranger}}

% \epigraph{\textit{``Every breath you take every move you make [...] I'll be watching you''}}{Sting and The Police, \textit{Every Breath You Take}}

\thispagestyle{empty}
\vspace*{\fill}
\setlength{\epigraphwidth}{0.7\linewidth}
\renewcommand{\epigraphflush}{center}
\renewcommand{\epigraphsize}{\large}
\epigraph{\textit{In dark times,\newline should the stars also go out?}}%
         {``Steban, the Student Communist''\\ \textit{Disco Elysium}}
\vfill

\begin{abstract}
    Respiratory motion is a significant problem when it comes to the quality of \gls{PET} images. Respiratory motion not only causes blurring of lesions, and other features within the lung, but can also cause problems when it comes to applying attenuation correction to the acquired \gls{PET} data. Respiratory motion can be mitigated by gating but this leads to longer acquisition time. Problems related to attenuation correction include the fact that the attenuation map is acquired temporally apart from the \gls{PET} scan, and different instructions are given to the patient for each scan (free breathing vs breath hold). Thus, the \gls{Mu-Map} may not correspond to any position in the \gls{PET} data and certainly won't correspond to all of them. Meaning that if a static \gls{Mu-Map} is used to apply attenuation correction not only will the respiratory motion exist to corrupt the data but so will the mismatch of the \gls{Mu-Map}.

    Motion correction methods exist which incorporate registration in order to attempt to improve upon non-motion corrected results. Often these methods involve separating the \gls{PET} data into bins, where the respiratory motion is minimised within each bin. However, because of the high level of noise and low spatial resolution of \gls{PET} data, few bins are often used, leading to a not insignificant amount of respiratory motion still being present in the resultant images. Additionally, this approach does not solve the mismatch of the \gls{Mu-Map}. Logically, the bin closest to the \gls{Mu-Map} could be used as the reference bin for registration but it is not guaranteed that this bin will be close to the position of the \gls{Mu-Map} and as such artefacts will remain in the final image. Furthermore, registration fails when used on dynamic \gls{PET} data, where the signal from the aorta, at early time points, often leads to misregistration. More complex motion correction methods exist, however, these methods, in general, tend to be more resource intensive in both the sense of computation time and computational resources.

    This thesis focuses on the development of a motion correction method, which seeks to rectify some of the issues above by using respiratory gated data in combination with motion modelling. Firstly, this thesis will present more preliminary results, where the bounds of the problem were found. This includes experiments to discover the effectiveness of different motion correction techniques (focusing on motion modelling) in the case of \gls{TOF} vs \gls{Non-TOF} \gls{PET} data, especially where a high number of bins are used. This thesis will also present work related to attempting to solve the mismatch of the \gls{Mu-Map} by deforming it to the position during the gates. Furthermore, work will also be presented which compares the effects of the reconstruction and \gls{MM} fitting process, including using \gls{MLACF}, to approximate attenuation correction, as well as fitting a \gls{MM} on coarsely binned data and applying it to finely binned data.
    
    Finally, this thesis will also present work on the problems of \gls{DD} \gls{SS} extraction methods applied to dynamic \gls{PET}. A \gls{SS} is imperative to the effectiveness of binning data as well as to \gls{MM} fitting. By presenting work relevant to \gls{SS} extraction from dynamic \gls{PET} this work potentially opens the motion correction methods presented previously to the application of dynamic \gls{PET}.

    This thesis will conclude with a critique of the work presented previously as well as a look to the future of how this work could be improved or further used.
\end{abstract}

\begin{impactstatement}
    \begin{quote}
        The impact of the work contained in this thesis is positive both in the sense that it benefits the academic community as well as ideally benefiting the public in general. A draw to working on a topic in medical imaging is that the work which is done has the potential to improve the lives of many people who are otherwise suffering. It would be egotistical to assume, directly, that this work will help anyone. However, even if only one small aspect contained in this thesis goes on to influence a similar aspect of clinical practise, then it has had a measurable benefit. Many people suffer from diseases which affect the thorax. Respiratory motion makes imaging of these regions more difficult, it can lead to inconclusive or false negative findings. Through the inclusion of a respiratory motion correction, ionising follow up scans could be avoided and illnesses can be treated more accurately earlier. The motion correction algorithms developed were always intended to be as easy to implement into a clinical workflow as possible. The motion correction algorithms worked with the data that was previously already available, as robustly as possible, in the least time possible. Furthermore, the dynamic \gls{SS} extraction work potentially makes motion correction available in situations where it was not previously.
        
        Jeremy Bentham (one of the founders of \gls{UCL} and the philosophy of utilitarianism) defined the utility of an object as follows ``That property in any object, whereby it tends to produce benefit, advantage, pleasure, good, or happiness ... [or] to prevent the happening of mischief, pain, evil, or unhappiness to the party whose interest is considered.''. By Bentham's definition, of the utility of an object, the work contained in this thesis has positive utility for the public. This is because, as stated previously, the work produces benefit, advantage, good, and happiness through the potential reduction in evil and unhappiness.

        Academically, the work presented in this thesis was always conducted, as much as it could be, to align with the philosophy of open source, free as in freedom/libre, and collaboration. Work has been published and presented at international conferences, both as oral and poster presentations, and work has been submitted (and intended to be submitted) to journals as papers. A number of suggested future research opportunities are also presented whereby the methodologies formulated here could be improved or combined with others to further impact academia and clinical practise.
    \end{quote}
\end{impactstatement}

\begin{acknowledgements}
    This research was supported by \gls{GE} Healthcare, the \gls{NIHR} \gls{UCLH} Biomedical Research Centre and the \gls{UCL} \gls{EPSRC} \gls{CDT} in \gls{i4health} grant (EP/L$016478$/$1$).
    
    The software used was partly produced by the \gls{CCP} on \gls{SyneRBI} \gls{UK} \gls{EPSRC} grant (EP/T$026693$/$1$).

    Firstly, I would like to thank my supervisors Kris Thielemans and Jamie McClelland. I am eternally grateful for the opportunity that you have provided to me, both to do a \gls{PhD} at all (especially at \gls{UCL}) as well as all of the support I have received throughout the (many) years that it has taken me to get to this point. I am most thankful for how patient you have been with me. It must have been frustrating for you, at times, but I hope you also got something out of it. I hope I didn't disappoint you.

    Next, I would like to thank all the people at \gls{GE} HealthCare for their insightful comments and help in formulating ideas, including Dylan (I hope you continue to enjoy basketball as much as you did last time when your team won), Scott Wollenweber, and Chuck Stearns.

    Potentially most importantly (sorry), I would like to write something about my dad. One of the reasons that I wanted to do a \gls{PhD} at all was because I wanted you to be proud of me. One of the reasons I did anything at all is because I wanted you to be proud of me. You didn't make it to the end of my \gls{PhD} and I have no idea what you would think now. On one hand I don't think you would have really cared that I did a \gls{PhD}, not because you don't care at all but because I didn't need to do a \gls{PhD} for you to be happy with what I am doing. I almost made the mistake of guessing what you wanted me to do while I was in college, luckily that sorted itself out quite well. I'm going to miss thinking about the fact that on the day I (hopefully) finally pass all of this you would have been so emotional, you were so emotional when I finished my undergraduate, never mind something more. I do not believe at all that you are anywhere other than in a box at the bottom of the stairs, or in my ring (or in Sam's ring, which he used as an extremely expensive method of scattering your ashes). I do not believe that you are watching over me or any other rubbish, like some people like to say. However, I do believe that I wouldn't be the person that I am today if you weren't my dad. I loved all the little things that you found interesting, I know way more about pitot tubes than anyone has any right to know. My interests in electronics and engineering are specifically because of you. I'm disappointed that you never tried to do anything more with what you could have obviously done. I think the best thing I can do for you now is to lead a good (and healthy) life. Thank you, I still miss you.

    Additionally, I would like to thank my mum, who in her own way, I know, has always wanted to show that she cares for me. I wish you all the best and I hope you can always be happy and proud of me. I would like to thank my brother, (little) Sam, I cherish and appreciate the very special relationship we have, you are one of my best friends. I will always be there for you. Finally, I would like to thank my granddad and my grandma. To my granddad, I appreciate how you always ask about and listened to me talking about my work, even though it must have been very boring at times. I wouldn't have been able to finish anything without you having read and commented on my work, which must have been very time consuming. To my grandma, thank you for showing an interest in my field of work, reading about it in magazines or watching it on the news. I really appreciate the effort you have made, especially because I know you have no interest in technology.

    I'd like to thank Nikos because without your supervision during my \gls{MSc} I would not even have thought it possible to do a \gls{PhD} at \gls{UCL}.

    I would like to thank all the people I have met during the time I was doing my \gls{PhD}. First, I'd like to thank Ludovica, you are one of the most loyal people I know. Even though we have had our differences sometimes (mostly caused by my own stupidity), you've never held it against me and have been (and still are) always there for me, both as a colleague and as one of my best friends. Next, I would like to thank Ashley for the times we have supported each other during our \glspl{PhD}, especially during the times we both doubted that we would be able to finish this. I especially appreciated and enjoyed being able to talk about working class things we both can relate to. I want to thank Ander, I appreciate all of your input for my work. I will never forget that you were the one who talked to me on the day my dad died. I hope you are happy now. I would like to thank Ade, you were the first person I met at \gls{UCL}, you showed me around and made me feel welcome and included. You're an incredibly nice and funny guy and I always enjoyed your company. I would like to thank Elise for all the help you gave me at the beginning of my \gls{PhD}.

    Not to forget, I would like to thank my friends Bousfield (big Sam), Jack, and Peter for simply being my friends. I have very fond memories of the time we have spent together, so far, and I am looking forward to all the times yet to come. If any of you see this, take a photograph, show me and I will buy you a beer, but only the first time. I would also like to thank Viv for helping me out of a sticky situation with my hand drawn figures. 

    Abschließend möchte ich mich bei Kathi bedanken. Ich würde gerne scherzen, dass es viel einfacher gewesen wäre, meine Doktorarbeit zu beenden, wenn Du nicht bei mir gewesen wärst, weil ich meine ganze Zeit nur mit Dir verbringen möchte. Aber das stimmt nicht (na ja, das mit der Zeit schon). Ich glaube nicht, dass ich meine Arbeit hätte abschließen können, zumindest nicht so glücklich, wie ich es bin, wenn Du nicht gewesen wärst. Obwohl in letzter Zeit so viel passiert ist, findest Du immer noch etwas in Dir, das Dir hilft, jeden Tag besser zu machen als den letzten. Ich glaube wirklich, dass du meine Arbeit öfter gelesen hast als ich selbst. Dafür möchte ich Dir danken. Ich denke, wir können mit Fug und Recht behaupten, dass wir ``in guten wie in schlechten Zeiten'' und ``in Krankheit und Gesundheit'' bereits gemeistert haben. Ich bin gespannt auf ``bis dass der Tod uns scheidet''.

    % Finally, I would like to thank Kathi. I wanted to joke that it would have been a lot easier to finish my thesis if you weren't around, because all I want to do is spend all my time on you. However, that's not true (well it is about the time thing, I can't help myself). I don't think I could have finished my thesis, at least not while being as happy as I am, if it wasn't for you. Even though there has been so much going on recently, you still find something in yourself that helps to make every day better than the last. I genuinely think you've read my thesis more times than I have. I really appreciate you. I think we can legitimately say that we've already completed ``for better, for worse'' and ``in sickness and in health'', I'm excited to see about ``till death do us part''. I'd also like to thank Birte for help with the German. Otherwise, with the German I know, the previous paragraph would have been mostly profanity.
    
    % \newpage
    
    % N\-9\-G\-s\-q\-H\-V\-Z\-3\-x\-l\-a\-t\-z\-d\-p\-i\-y\-e\-r\-q\-9\-8\-q\-l\-i\-G\-G\-V\-u\-p\-x\-1\-A\-F\-a\-u\-7\-9\-2\-9\-l\-l\-V\-j\-I\-N\-C\-u\-E\-F\-7\-O\-w\-o\-y\-L\-0\-e\-C\-Y\-B\-O\-0\-L\-V\-V\-4\-5\-o\-7\-y\-0\-a\-R\-W\-o\-d\-C\-d\-x\-N\-o\-n\-b\-5\-P\-V\-H\-4\-0\-k\-C\-z\-Q\-p\-a\-S\-H\-b\-4\-8\-f\-F\-t\-O\-V\-3\-M\-L\-I\-g\-t\-X\-c\-F\-V\-e\-x\-g\-P\-L\-Z\-1\-q\-l\-8\-7\-o\-n\-a\-N\-F\-a\-A\-F\-F\-C\-K\-p\-i\-7\-b\-1\-S\-e\-4\-R\-s\-I\-A\-6\-L\-t\-6\-N\-m\-k\-q\-w\-4\-W\-c\-h\-P\-9\-T\-a\-T\-I\-N\-u\-B\-L\-A\-H\-3\-2\-6\-s\-H\-Z\-4\-g\-z\-F\-X\-n\-J\-6\-s\-1\-7\-w\-k\-Z\-e\-I\-W\-1\-j\-x\-5\-M\-y\-8\-x\-Y\-H\-S\-F\-e\-g\-U\-R\-V\-K\-z\-l\-t\-x\-S\-F\-K\-E\-m\-E\-V\-h\-s\-x\-a\-w\-M\-C\-G\-8\-p\-z\-7\-c\-8\-o\-1\-E\-L\-o\-H\-L\-m\-p\-x\-V\-P\-U\-l\-F\-M\-d\-c\-i\-v\-0\-Y\-1\-H\-E\-h\-t\-p\-Y\-j\-t\-1\-7\-s\-j\-p\-h\-q\-w\-0\-Q\-Y\-w\-V\-b\-M\-B\-N\-m\-F\-B\-h\-k\-g\-P\-i\-6\-y\-N\-J\-k\-p\-G\-C\-U\-l\-9\-s\-V\-K\-h\-V\-P\-5\-3\-W\-p\-Y\-u\-F\-E\-O\-l\-t\-H\-o\-8\-4\-K\-J\-L\-W\-e\-O\-h\-C\-m\-J\-B\-T\-c\-G\-a\-j\-n\-l\-A\-u\-D\-x\-L\-P\-w\-9\-u\-n\-8\-k\-q\-P\-6\-h\-B\-H\-Y\-q\-L\-5\-w\-n\-2\-b\-Q\-l\-0\-p\-q\-M\-x\-y\-2\-5\-1\-B\-f\-R\-G\-y\-L\-3\-5\-O\-3\-t\-q\-j\-Z\-I\-Q\-9\-w\-f\-4\-H\-s\-J\-r\-i\-u\-9\-F\-p\-O\-Y\-4\-E\-Y\-o\-T\-G\-f\-p\-j\-m\-6\-G\-t\-v\-o\-W\-a\-i\-K\-x\-1\-J\-J\-n\-W\-F\-f\-7\-8\-S\-n\-i\-f\-T\-0\-u\-b\-A\-1\-W\-4\-7\-X\-K\-0\-0\-m\-D\-p\-1\-n\-o\-s\-a\-3\-C\-l
    
    % 1966 1995
\end{acknowledgements}

\preto\fullcite{\AtNextCite{\defcounter{maxnames}{99}}}

\begin{publications}
    \begin{itemize}
        \item \fullcite{Whitehead2023DataComponents} (In Review)
    
        \item \fullcite{Shahin2023CenTime:Analysis} (Accepted)

        \item (Oral presentation) \fullcite{Whitehead2023APET} (Accepted)

        \item (Poster presentation) \fullcite{Whitehead2023DeepPatients} (Accepted)
    
        \item \fullcite{Ovtchinnikov2023SIRFFramework}
    
        \item \fullcite{Brusaferri2022DetectionData}

        \item (Poster presentation) \fullcite{Whitehead2022PET/CTData} (Accepted)

        \item (Oral presentation) \fullcite{Whitehead2022DataPCA} (Accepted)

        \item (Poster presentation) \fullcite{Whitehead2022Pseudo-BayesianPET} (Accepted)
     
        \item \fullcite{Ferrante2022PhysicallyImaging}

        \item \fullcite{Pasca2022SIRF-SuperBuild}
    
        \item (Poster presentation) \fullcite{Whitehead2021ComparisonMap}
    
        \item \fullcite{Biguri2021SystematicPET}

        \item \fullcite{Ovtchinnikov2021SIRFMachine}
     
        \item (Oral presentation) \fullcite{Whitehead2020PET/CTFields}
     
        \item (Poster presentation) \fullcite{Whitehead2019ImpactPET}
    
        \item \fullcite{Efthimiou2019PreliminaryPET}
    \end{itemize}
\end{publications}

\tableofcontents
\listoffigures
\listoftables
\printglossaries
